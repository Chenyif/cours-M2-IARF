%%%% patron de format latex pour rfia 2000.
%%%% sans garanties. Plaintes \`a envoyer \`a \dev\null.
%%%% deux colonnes pas de num\'erotation et 10 points
%%%% necessite les fichiers a4.sty french.sty et rfia2000.sty

%%%% Pour \LaTeXe
\documentclass[a4paper,twoside,french]{article}
\usepackage{rfia2000}
\usepackage[T1]{fontenc}
\usepackage{babel}
\usepackage{times}

%%%% Pour \LaTeXe sans babel
%%%% \documentclass[a4paper,twoside]{article}
%%%% \usepackage{rfia2000}
%%%% \usepackage{french}
%%%% \usepackage{times}

%%%% Pour \LaTeX remplacer les trois ligne pr\'ec\'edente par les deux
%%%% suivantes
%%%%\documentstyle[a4,french]{article}
%%%%%%%% feuile de style LaTeX pour RFIA'2000.
%%%%
%%%% usage: copier ce fichier sous le nom rfia2000.sty sous votre
%%%% repertoire courant. La source du fichier LaTeX doit debuter par
%%%%
%%%% 1/ pour LaTeX2e:
%%%%      \documentclass[a4paper]{article}
%%%%      \usepackage{rfia2000}
%%%%      \usepackage{french}
%%%%
%%%% 2/ pour LaTeX:
%%%%      \documentstyle[a4paper,french]{article}
%%%%      %%%% feuile de style LaTeX pour RFIA'2000.
%%%%
%%%% usage: copier ce fichier sous le nom rfia2000.sty sous votre
%%%% repertoire courant. La source du fichier LaTeX doit debuter par
%%%%
%%%% 1/ pour LaTeX2e:
%%%%      \documentclass[a4paper]{article}
%%%%      \usepackage{rfia2000}
%%%%      \usepackage{french}
%%%%
%%%% 2/ pour LaTeX:
%%%%      \documentstyle[a4paper,french]{article}
%%%%      %%%% feuile de style LaTeX pour RFIA'2000.
%%%%
%%%% usage: copier ce fichier sous le nom rfia2000.sty sous votre
%%%% repertoire courant. La source du fichier LaTeX doit debuter par
%%%%
%%%% 1/ pour LaTeX2e:
%%%%      \documentclass[a4paper]{article}
%%%%      \usepackage{rfia2000}
%%%%      \usepackage{french}
%%%%
%%%% 2/ pour LaTeX:
%%%%      \documentstyle[a4paper,french]{article}
%%%%      \input{rfia2000}

\twocolumn
\sloppy
\flushbottom
\parindent 0em
\leftmargini 2em
\leftmarginv .5em
\leftmarginvi .5em
\marginparwidth 20pt
\marginparsep 10pt
\oddsidemargin -0.77cm
\evensidemargin -0.77cm

\topmargin      0mm
\headheight     0mm
\headsep        1mm
%\evensidemargin 0mm
%\oddsidemargin  0mm
\textheight     9.4in
\textwidth      6.875in
\columnsep      1cm

\pagestyle{empty}
\makeatletter
\def\@normalsize{\@setsize\normalsize{10pt}\xpt\@xpt
\abovedisplayskip 10pt plus2pt minus5pt\belowdisplayskip \abovedisplayskip
\abovedisplayshortskip \z@ plus3pt\belowdisplayshortskip 6pt plus3pt
minus3pt\let\@listi\@listI}
\def\section{\@startsection {section}{1}{\z@}
        {-1ex plus -.2ex minus -.2ex}{1ex plus .2ex}{\Large\bf}}
\def\subsection{\@startsection {subsection}{2}{\z@}
        {-1ex plus -.2ex minus -.2ex}{1ex plus .2ex}{\large\bf}}

        
        

\renewcommand{\subsubsection}[1]{\paragraph*{\bf #1.}}
\def\paragraph{\@startsection {paragraph}{4}{\z@}
        {-1ex plus -.2ex minus -.2ex}{-1em}{\normalsize\bf}}
\def\subparagraph{\@startsection {subparagraph}{4}{\parindent}
        {-1ex plus -.2ex minus -.2ex}{-1em}{\normalsize\bf}}
\makeatother




\twocolumn
\sloppy
\flushbottom
\parindent 0em
\leftmargini 2em
\leftmarginv .5em
\leftmarginvi .5em
\marginparwidth 20pt
\marginparsep 10pt
\oddsidemargin -0.77cm
\evensidemargin -0.77cm

\topmargin      0mm
\headheight     0mm
\headsep        1mm
%\evensidemargin 0mm
%\oddsidemargin  0mm
\textheight     9.4in
\textwidth      6.875in
\columnsep      1cm

\pagestyle{empty}
\makeatletter
\def\@normalsize{\@setsize\normalsize{10pt}\xpt\@xpt
\abovedisplayskip 10pt plus2pt minus5pt\belowdisplayskip \abovedisplayskip
\abovedisplayshortskip \z@ plus3pt\belowdisplayshortskip 6pt plus3pt
minus3pt\let\@listi\@listI}
\def\section{\@startsection {section}{1}{\z@}
        {-1ex plus -.2ex minus -.2ex}{1ex plus .2ex}{\Large\bf}}
\def\subsection{\@startsection {subsection}{2}{\z@}
        {-1ex plus -.2ex minus -.2ex}{1ex plus .2ex}{\large\bf}}

        
        

\renewcommand{\subsubsection}[1]{\paragraph*{\bf #1.}}
\def\paragraph{\@startsection {paragraph}{4}{\z@}
        {-1ex plus -.2ex minus -.2ex}{-1em}{\normalsize\bf}}
\def\subparagraph{\@startsection {subparagraph}{4}{\parindent}
        {-1ex plus -.2ex minus -.2ex}{-1em}{\normalsize\bf}}
\makeatother




\twocolumn
\sloppy
\flushbottom
\parindent 0em
\leftmargini 2em
\leftmarginv .5em
\leftmarginvi .5em
\marginparwidth 20pt
\marginparsep 10pt
\oddsidemargin -0.77cm
\evensidemargin -0.77cm

\topmargin      0mm
\headheight     0mm
\headsep        1mm
%\evensidemargin 0mm
%\oddsidemargin  0mm
\textheight     9.4in
\textwidth      6.875in
\columnsep      1cm

\pagestyle{empty}
\makeatletter
\def\@normalsize{\@setsize\normalsize{10pt}\xpt\@xpt
\abovedisplayskip 10pt plus2pt minus5pt\belowdisplayskip \abovedisplayskip
\abovedisplayshortskip \z@ plus3pt\belowdisplayshortskip 6pt plus3pt
minus3pt\let\@listi\@listI}
\def\section{\@startsection {section}{1}{\z@}
        {-1ex plus -.2ex minus -.2ex}{1ex plus .2ex}{\Large\bf}}
\def\subsection{\@startsection {subsection}{2}{\z@}
        {-1ex plus -.2ex minus -.2ex}{1ex plus .2ex}{\large\bf}}

        
        

\renewcommand{\subsubsection}[1]{\paragraph*{\bf #1.}}
\def\paragraph{\@startsection {paragraph}{4}{\z@}
        {-1ex plus -.2ex minus -.2ex}{-1em}{\normalsize\bf}}
\def\subparagraph{\@startsection {subparagraph}{4}{\parindent}
        {-1ex plus -.2ex minus -.2ex}{-1em}{\normalsize\bf}}
\makeatother




\begin{document}
%%%%%Pas de date
\date{}
%%%%% Titre gras 14 points
\title{\Large\bf Mon merveilleux article pour RFIA'2012
       }
%%%%% Si auteur unique
%\author{L. Auteur \\
%%  Son institut \\
%%  Son addresse \\
%%  Son email}
%%%% pour deux auteurs
\author{\begin{tabular}[t]{c@{\extracolsep{8em}}c}
%%%% pour trois auteurs
%%%%\author{\begin{tabular}[t]{c@{\extracolsep{6em}}c@{\extracolsep{6em}}c}
%%%% pour quatre auteurs
%%%%\author{\begin{tabular}[t]{c@{\extracolsep{4em}}c@{\extracolsep{4em}}c@{\extracolsep{4em}}c}
%%%%pour plus d\'ebrouillez-vous !
M. Oim\^eme${}^1$  & M. Oncopain${}^2$ \\
\end{tabular}
{} \\
 \\
${}^1$        Mon Institut   \\
${}^2$        Son Institut
{} \\
 \\
Mon adresse compl\`ete \\
Mon adresse \'electronique
}
\maketitle
%%%%  Pas de num\'erotation sur la page de titre
\thispagestyle{empty}
\subsection*{R\'esum\'e}
{\em
C'est mon r\'esum\'e. Il doit occuper quelques centim\`etres (moins de 10)
y compris son titre qui est en 12 points gras.
}
\subsection*{Mots Clef}
Exemple type, format, mod�le.

\subsection*{Abstract}
{\em
It's the English version of the abstract. Exactly as in French it must
be short. It must speak of the same topics...
}
\subsection*{Keywords}
Example, model, template.

\section{Introduction}
Le contenu de l'article peut \^etre r\'edig\'e avec n'importe quel formateur
ou traitement de texte, pourvu qu'il r\'eponde aux crit\`eres de
pr\'esentation donn\'es ici. L'objectif vis\'e est de proposer une unit\'e
de pr\'esentation des actes, et nous vous invitons \`a tenter de respecter
ce mod\`ele autant que le permet votre logiciel favori.

Pour les auteurs utilisant \LaTeX\, le fichier source de ce
texte (\verb|latex.tex|) est lui-m\^eme une base pour obtenir une
sortie conforme avec \LaTeXe\footnote{ou \LaTeX}.

Les autres trouveront des renseignements (peut-\^etre) plus lisibles
pour eux dans le fichier \verb|word.doc| (Word).
  
Un fichier PDF contient \'egalement la d\'efinition compl\`ete de la
pagination. Il a pour nom \verb|pages.pdf| et est cliquable sur la page
Web.

La base du texte est du Times-Roman 10 points pr\'esent\'e en deux colonnes.
La s\'eparation inter-colonne est de 1 cm.

Le titre principal est en 14 points gras (28 points = 1cm).

Dans les sections, le titre est en 12 points gras.
Les paragraphes ne sont pas decal\'es.

Les sous-sections num\'erot\'ees comme suit~:
\subsection{Travaux ant\'erieurs}
Les en-t\^etes sont \'egalement en 12 points gras.

Il n'y a pas n\'ecessairement d'espacement entre les paragraphes.

Les r\'ef\'erences \`a la Bibliographie peuvent \^etre de la forme
 \cite{key:foo} o\`u \cite{foo:baz}.  Les num\'eros correspondent \`a
 l'ordre d'apparition dans la bibliographie, pas dans le texte.
 L'ordre alphab\'etique est conseill\'e.

\section{Le coin \LaTeX}
Pour les utilisateurs de \LaTeX,
ce patron est minimaliste et vous aurez besoin de votre manuel \LaTeX
pour ins\'erer \'equations et images.

Pour les images le << paquet >> \verb|graphicx| est tr\`es bien.

Les fichiers de style n\'ecessaires pour la compilation \LaTeX\ sont :
\begin{itemize}
  \item \verb|a4.sty|  (pour \LaTeX, mais pas \LaTeXe)
  \item \verb|french.sty| (ou Babel fran\c{c}ais)
  \item \verb|rfia2000.sty|
\end{itemize}
Vous devriez avoir les deux premiers dans votre instalation de
\LaTeX. Le dernier contient la d\'efinition des marges et vous devrez le
r\'ecup\'erer.

\subsection*{Annexe}
Merci de votre participation.

\begin{thebibliography}{9}
\bibitem{foo:baz}
U. Nexpert,
{\em Le livre,}
Son Editeur, 1929.
\bibitem{key:foo}
I. Troiseu-Pami,
 Un article int\'eressant,
{\em Journal de Spirou}, Vol. 17, pp. 1-100, 1987.
\end{thebibliography}
\end{document}


